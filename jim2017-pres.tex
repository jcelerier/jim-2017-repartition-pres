\documentclass[draft,handout]{beamer}

\usepackage{polyglossia}
\setdefaultlanguage{french}

\setmainfont{Tribun ADF Std}
\setsansfont{Tribun ADF Std}
\setmonofont{Fira Code}

\usepackage{soul}
\usepackage{microtype}
\usepackage{csquotes}
\usepackage{default}
\usepackage{hyperref}
\hypersetup{colorlinks=true}
\usetheme{simple}
\usepackage{graphicx}
\usepackage[justification=centering]{caption}
\usepackage{subcaption}
\usepackage{listings}
\usepackage{pstricks}
\captionsetup[subfigure]{labelformat=empty}
\captionsetup[figure]{labelformat=empty}
\setbeamertemplate{caption}{\raggedright\insertcaption\par}
\setbeamerfont{frametitle}{size=\LARGE}
\newfontfamily\DejaSans{DejaVu Sans}
\setbeamerfont{title}{family=\texttt,size=\huge}
\usepackage[scale=2]{ccicons}
\newfontfamily\unicodefun[Ligatures=TeX]{Symbola}
\newfontfamily\unicodefun{Droid Sans}


\lstdefinelanguage{JavaScript}{
    keywords={typeof, new, true, false, catch, function, return, null, catch, switch, var, if, in, while, do, else, case, break},
    keywordstyle=\color{blue}\bfseries,
    ndkeywords={class, export, boolean, throw, implements, import, this},
    ndkeywordstyle=\color{darkgray}\bfseries,
    identifierstyle=\color{black},
    sensitive=false,
    comment=[l]{//},
    morecomment=[s]{/*}{*/},
    commentstyle=\color{purple}\ttfamily,
    stringstyle=\color{red}\ttfamily,
    morestring=[b]',
    morestring=[b]"
}

\lstset{
    language=JavaScript,
    extendedchars=true,
    basicstyle=\footnotesize\ttfamily,
    showstringspaces=false,
    showspaces=false,
    numberstyle=\footnotesize,
    numbersep=9pt,
    tabsize=2,
    breaklines=true,
    showtabs=false,
    captionpos=b
}

\title{Exécution répartie de partitions interactives}
\subtitle{}
\date{}
\author{Jean-Michaël Celerier$\mathsf{^{1,2}}$~\\ Myriam Desainte-Catherine$\mathsf{^{2}}$~\\ Jean-Michel Couturier$\mathsf{^{1}}$}
\institute{1. Blue Yeti --- 2. SCRIME / LaBRI}
\titlegraphic{
    \includegraphics[width=0.2\textwidth]{images/logobl.png} \vspace{1cm}
    \includegraphics[width=0.5\textwidth]{images/scrime.jpg}}
\usepackage{tikz}

\newsavebox{\codebox}% For storing listings
\begin{document}
\maketitle
\begin{frame}
    \tableofcontents
\end{frame}

\section{Introduction}
\begin{frame}
    \frametitle{Problématique}
    \Large
\end{frame}

\begin{frame}
    \frametitle{Quarrè}
    \Large
\end{frame}

\begin{frame}
    \frametitle{Existant}
    \begin{itemize}
        \item Synchronisation d'horloge
        \item Serveurs de son
        \item Écriture répartie
    \end{itemize}
\end{frame}

\begin{frame}
\frametitle{Rappel d'i-score}
\end{frame}

\section{Répartition}


\subsection{Groupes}
\begin{frame}
\frametitle{Groupes}
{\Large Comment assurer l'indépendance vis-à-vis du matériel lors de l'écriture ?}
\end{frame}


\subsection{Répartition des contenus}
\begin{frame}
\frametitle{Répartition des contenus}
{\Large Pour un agencement de structures temporelles donné, quelles sont les exécutions pouvant être définies ?}

\begin{itemize}
    \item Libre: chaque machine exécute indépendamment.
    \item Partagée: les temporalités sont identiques, les contenus changent.
    \item Mixte: Les temporalités sont identiques au sein d'un groupe.
\end{itemize}
\end{frame}

\begin{frame}
\frametitle{Exécution libre}
\begin{figure}
    \centering
    \begin{tikzpicture}
    \fill (0, 0.141497) circle (0.075) ; % State.0 
\fill (0, 0.141393) circle (0.075) ; % State.1 
\fill (3.02682, 0.141393) circle (0.075) ; % State.2 
\fill (5, 0.141393) circle (0.075) ; % State.4 
\draw[line width=1pt] (0, 0.141497)  -- (0, 0.141393) ; % TimeNode.0 
\draw[line width=1pt] (3.02682, 0.141393)  -- (3.02682, 0.141393) ; % whim75dash50 
\draw[line width=1pt] (5, 0.141393)  -- (5, 0.141393) ; % them13ergo22 
\draw[line width=1pt] (0, 0.141393)  -- (1.69361, 0.141393) ; % wavy46flak3 
\draw[dashed,line width=1pt] (1.69361, 0.141393)  -- (3.02682, 0.141393) ; % wavy46flak3 
\draw[line width=0.7pt] (1.93361, 0.294393) arc(90:270:0.15) ; % wavy46flak3 
\draw[line width=1pt] (3.02682, 0.141393)  -- (5, 0.141393) ; % dock4chew40 

    \end{tikzpicture}
    \caption{Notation}
\end{figure}
\begin{figure}
    \centering
    \begin{tikzpicture}
    \fill (0, 0.0104813) circle (0.075) ; % State.0 
\fill (0, 0.0104807) circle (0.075) ; % State.1 
\fill (2.68116, 0.0104807) circle (0.075) ; % State.2 
\fill (5, 0.0104807) circle (0.075) ; % State.4 
\draw[line width=1pt] (0, 0.0104813)  -- (0, 0.0104807) ; % TimeNode.0 
\draw[line width=1pt] (2.68116, 0.0104807)  -- (2.68116, 0.0104807) ; % whim75dash50 
\draw[line width=1pt] (5, 0.0104807)  -- (5, 0.0104807) ; % them13ergo22 
\draw[line width=1pt] (0, 0.0104807)  -- (2.68116, 0.0104807) ; % wavy46flak3 
\draw[line width=1pt] (2.68116, 0.0104807)  -- (5, 0.0104807) ; % dock4chew40 

    \end{tikzpicture}
    \caption{Déroulement sur la machine 1}
\end{figure}
\begin{figure}
    \centering
    \begin{tikzpicture}
    \fill (0, 0.0104814) circle (0.075) ; % State.0 
\fill (0, 0.0104808) circle (0.075) ; % State.1 
\fill (3.38811, 0.0104808) circle (0.075) ; % State.2 
\fill (5, 0.0104808) circle (0.075) ; % State.4 
\draw[line width=1pt] (0, 0.0104814)  -- (0, 0.0104808) ; % TimeNode.0 
\draw[line width=1pt] (3.38811, 0.0104808)  -- (3.38811, 0.0104808) ; % whim75dash50 
\draw[line width=1pt] (5, 0.0104808)  -- (5, 0.0104808) ; % them13ergo22 
\draw[line width=1pt] (0, 0.0104808)  -- (3.38811, 0.0104808) ; % wavy46flak3 
\draw[line width=1pt] (3.38811, 0.0104808)  -- (5, 0.0104808) ; % dock4chew40 

    \end{tikzpicture}
    \caption{Déroulement sur la machine 2}
\end{figure}
\end{frame}
\begin{frame}
\frametitle{Exécution partagée}
\begin{figure}
    \centering
    \begin{tikzpicture}
    
\draw[line width=1pt] (0, 0.655491)  -- (0, 0.653257) ; % TimeNode.0 
\draw[line width=1pt] (1.26282, 0.655491)  -- (1.26282, 0.655491) ; % whim75dash50 
\draw[line width=1pt] (2.94308, 0.655491)  -- (2.94308, 0.655491) ; % them13ergo22 
\draw[line width=1pt,color=cyan] (0, 0.655491)  -- (0.138662, 0.655491) ; % wavy46flak3 
\draw[dashed,line width=1pt,color=cyan] (0.138662, 0.655491)  -- (1.26282, 0.655491) ; % wavy46flak3 
\draw[line width=0.7pt,color=cyan] (0.378662, 0.808491) arc(90:270:0.15) ; % wavy46flak3 
\draw[line width=1pt,color=cyan] (0, 0.555491)  -- (1.26282, 0.555491)  -- (1.26282, -0.444509)  -- (0, -0.444509)  -- (0, 0.555491) ;
\draw[line width=1pt,color=cyan] (0, -0.444509)  -- (1.26282, 0.555491) ;
\draw[line width=1pt,color=orange] (1.26282, 0.655491)  -- (2.94308, 0.655491) ; % dock4chew40 
\draw[line width=1pt,color=orange] (1.26282, 0.555491)  -- (2.94308, 0.555491)  -- (2.94308, -0.444509)  -- (1.26282, -0.444509)  -- (1.26282, 0.555491) ;
\draw[line width=1pt,color=orange] (1.26282, -0.444509)  -- (2.94308, 0.555491) ;

\fill (0, 0.65508) circle (0.075) ; % State.0 
\fill (0, 0.655491) circle (0.075) ; % State.1 
\fill (1.26282, 0.655491) circle (0.075) ; % State.2 
\fill (2.94308, 0.655491) circle (0.075) ; % State.4 
\fill (0, 0.653257) circle (0.075) ; % State.5 
    \end{tikzpicture}
    \caption{Notation}
\end{figure}
\begin{figure}
    \centering
    \begin{tikzpicture}
    
\draw[line width=1pt] (0, 0.0104813)  -- (0, 0.0104807) ; % TimeNode.0 
\draw[line width=1pt] (3.30592, 0.0104807)  -- (3.30592, 0.0104807) ; % whim75dash50 
\draw[line width=1pt] (5, 0.0104807)  -- (5, 0.0104807) ; % them13ergo22 
\draw[line width=1pt] (0, 0.0104807)  -- (3.30592, 0.0104807) ; % wavy46flak3 
\draw[line width=1pt,color=orange] (3.30592, 0.0104807)  -- (5, 0.0104807) ; % dock4chew40 
\draw[line width=1pt,color=orange] (3.30592, -0.0895193)  -- (5, -0.0895193)  -- (5, -1.08952)  -- (3.30592, -1.08952)  -- (3.30592, -0.0895193) ;
\draw[line width=1pt,color=orange] (3.30592, -1.08952)  -- (5, -0.0895193) ;

\fill (0, 0.0104813) circle (0.075) ; % State.0 
\fill (0, 0.0104807) circle (0.075) ; % State.1 
\fill (3.30592, 0.0104807) circle (0.075) ; % State.2 
\fill (5, 0.0104807) circle (0.075) ; % State.4 
    \end{tikzpicture}
    \caption{Déroulement sur la machine 1}
\end{figure}
\begin{figure}
    \centering
    \begin{tikzpicture}
    \input{scores/shared-3.tex}
    \end{tikzpicture}
    \caption{Déroulement sur la machine 2}
\end{figure}
\end{frame}
\begin{frame}
\frametitle{Exécution mixte}
\end{frame}

\subsection{Synchronisation}
\begin{frame}
\frametitle{Synchronisation}
\end{frame}

\begin{frame}
\frametitle{Compensation de latence}
\end{frame}

\begin{frame}
\frametitle{Consensus et interaction}
\end{frame}

\section{Lien avec les objets d'i-score}
\begin{frame}
\frametitle{Lien avec les objets d'i-score}
* Point d'interaction
* Condition
* Vitesse d'exécution
\end{frame}

\section{Conclusion}
\begin{frame}
    \frametitle{Conclusion}  
    \Large
    \begin{itemize}
        \item<1> tata
        \item<2> tutu
        \item<3> toto
    \end{itemize}
\end{frame}

\begin{frame}[allowframebreaks]%in case more than 1 slide needed
    
    %remove the icon
    \setbeamertemplate{bibliography item}{}
    
    %remove line breaks
    \setbeamertemplate{bibliography entry title}{}
    \setbeamertemplate{bibliography entry location}{}
    \setbeamertemplate{bibliography entry note}{}
    
    {\footnotesize
        \nocite{*}
        \bibliographystyle{IEEEtran}
        \bibliography{icmc2016}
    }
\end{frame}

\begin{frame}
    \frametitle{Liens} 
    \Large
    \begin{itemize}
        \setlength\itemsep{1em}
        \item \textbf{i-score} : \url{www.i-score.org}
    \end{itemize}
        
    \centering
    \vspace{2em}
    \Large{Merci ! Des questions ?}
    \vspace{2em}
    
    \small{Remerciements : Serge Chaumette, Pierre Cochard}
    
    \vspace{1em}
    
    \tiny{Utilise le thème Beamer 'simple' theme de (Facundo Muñoz) ainsi que les fontes Fira de Mozilla}
\end{frame}
\end{document}
